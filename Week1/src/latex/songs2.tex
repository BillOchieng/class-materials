% Comments are added in Latex using the percentage sign. 
\documentclass{article}
\usepackage[utf8]{inputenc}
\usepackage{verbatim}
\usepackage{fancyvrb}
\usepackage{graphicx}
\usepackage{latexsym}
\usepackage{color}
\usepackage{listings}

\usepackage{algorithm2e}
\usepackage{algorithmic}

\title{Songs}
\author{BOchieng}
\date{January 20th 2023}

\begin{document}
\begin{itemize}
\item[]
\textbf{Algorithm:} Find Least Played Song(S)
\item[]
\textbf{Input:} A set of play counts associated with a variety of songs inside a playlist.
\item[]
\textbf{Output:} The least played song.
\item[]
\scalebox{1}{
\begin{algorithm}[H]
\begin{algorithmic}[1]
% add your algorithm here
% please note: this code may not compile without adding the algorithmic logic in here ...
\STATE $temp \gets S[0]$ %statement which is temp
\STATE $res \gets 1$ % is the arrow for assign
\FOR{$i=1$ to $|S|$} % the dollar sign is for any math computation/ start of for loop
\IF{$S[i] > temp$} % open if
\STATE $temp \gets S[i]$
\STATE $res \gets i-1$
\ENDIF % end if 
\ENDFOR % for loop end 
\STATE \textbf{return} res; % return statement

\end{algorithmic}
\label{alg:seq}
\end{algorithm}
}
\end{itemize}
\end{document}
